% \iffalse meta-comment
% 
% Copyright (C) 2018 by M.J.W. Snippe
% -----------------------------------
% 
% This work may be distributed and/or modified under the
% conditions of the LaTeX Project Public License, either version 1.3
% of this license or (at your option) any later version.
% The latest version of this license is in
% 
%     http://www.latex-project.org/lppl.txt
% 
% and version 1.3 or later is part of all distributions of LaTeX
% version 2005/12/01 or later.
% 
% This work has the LPPL maintenance status `maintained'.
% 
% The Current Maintainer of this work is M.J.W. Snippe
% 
% This work consists of the files found at https://github.com/MaxSnippe/xbondgraphs.
% 
% \fi
% 
% \iffalse
%<package>\NeedsTeXFormat{LaTeX2e}[2018/04/01]
%<package>\ProvidesPackage{xbondgraphs}
%<package>    [2018/05/02 v0.0.1 Bond graph drawing using TikZ]
% 
%<*driver>
\documentclass{ltxdoc}
\usepackage{xbondgraphs}
\usepackage{xspace}
\usepackage{mathpazo}
\usepackage{pifont}
\usepackage{iconix}
\usepackage{booktabs}
\usepackage{tabu}
\usepackage{longtable}
\usepackage{footnote}
\usepackage{subcaption}
\usepackage{listings}
\usepackage[numbered]{hypdoc}
\usepackage[noabbrev]{cleveref}
\usepackage{lstautogobble}
% 
\pgfmathsetlengthmacro{\multibondwidth}{30mm}
\pgfmathsetlengthmacro{\singlebondwidth}{12.5mm}
\pgfmathsetmacro{\barbangle}{40}
%
\hyphenation{bond-graphs x-bond-graphs}
\newcommand\textvtt[1]{{\normalfont\fontfamily{cmvtt}\selectfont #1}}
%
\makesavenoteenv{table}
%
\lstset{
	autogobble,
	tabsize=4,
	breaklines=true,
	basicstyle=\ttfamily\small,
	language=[LaTeX]Tex,
}
%
\usetikzlibrary{positioning}
%
\newcommand{\Tikz}{Ti\textit{k}Z\xspace}
\newcommand{\pgf}{PGF\xspace}
\newcommand{\xbondgraphs}{\textsf{xbondgraphs}\xspace}
\newcommand{\bondgraphs}{\textsf{bondgraphs}\xspace}
\let\origLaTeX\LaTeX
\def\LaTeX{\origLaTeX\xspace}
\def\cmark{\ding{51}}
\def\xmark{\ding{55}}
%
\captionsetup{labelfont=bf,font=small,labelsep=endash}
\captionsetup[subfigure]{font=footnotesize}
%
\tikzset{
	> = stealth,
	arrow tip demonstration/.style={red,opacity=.75},
	arrow tip line style/.style={gray!40!black,line width = 0.5pt},
	annotation style/.style={line width = .4pt,gray,text=black,<->},
}
%
\EnableCrossrefs
\CodelineIndex
\RecordChanges
\begin{document}
	\DocInput{xbondgraphs.dtx}
\end{document}
%</driver>
% \fi
%
% \CheckSum{0}
%
% \CharacterTable
%  {Upper-case \A\B\C\D\E\F\G\H\I\J\K\L\M\N\O\P\Q\R\S\T\U\V\W\X\Y\Z
%   Lower-case \a\b\c\d\e\f\g\h\i\j\k\l\m\n\o\p\q\r\s\t\u\v\w\x\y\z
%   Digits \0\1\2\3\4\5\6\7\8\9
%   Exclamation \! Double quote \" Hash (number) \#
%   Dollar \$ Percent \% Ampersand \&
%   Acute accent \’ Left paren \( Right paren \)
%   Asterisk \* Plus \+ Comma \,
%   Minus \- Point \. Solidus \/
%   Colon \: Semicolon \; Less than \<
%   Equals \= Greater than \> Question mark \?
%   Commercial at \@ Left bracket \[ Backslash \\
%   Right bracket \] Circumflex \^ Underscore \_
%   Grave accent \‘ Left brace \{ Vertical bar \|
%   Right brace \} Tilde \~}
%
% \changes{v0.0.1}{2018/05/02}{Initial version}
%
% \GetFileInfo{xbondgraphs.sty}
%
% \DoNotIndex{\#,\$,\%,\&,\@,\\,\{,\},\^,\_,\~,\ }
% \DoNotIndex{\if,\else,\fi,\def,\ifcase,\or}
%
% \title{\xbondgraphs\thanks{This document corresponds to \xbondgraphs{}\space\fileversion, dated\space\filedate.} -- drawing bond graphs using \Tikz}
% \author{Marcus J.W. Snippe\thanks{E-mail: \href{mailto:m.j.w.snippe@saxion.nl}{m.j.w.snippe@saxion.nl}}}
%
% \maketitle
%
% \begin{abstract}
% When using the \xbondgraphs-package, the user is able to draw visually pleasing bond graphs\footnote{\url{https://en.wikipedia.org/wiki/Bond_graph}}, while mostly maintaining the standard notation of \Tikz drawings. It defines two new \pgf arrows, an accompanying decoration to ensure the direction of the barb, as well as a \pgf shape for power (de-)mux elements. This package is based on the \bondgraphs package by G. Folkertsma\footnote{\url{https://ctan.org/pkg/bondgraphs}}, but does not (yet) cover all its functions. It \emph{might} result in more appealing bond graphs.
% \end{abstract}