\iffalse
% !TEX root = xbondgraphs.dtx
\fi

\maketitle

\begin{abstract}
    Using the \xbondgraphs-package, one can draw visually pleasing bond graphs, while mostly maintaining the standard notation of \Tikz drawings. It defines two new \pgf arrows, a \pgf decoration to ensure the direction of the barb, as well as a \pgf shape for power (de-)mux elements. This package is based on the \bondgraphs package by Geert Folkertsma\footnote{\url{https://ctan.org/pkg/bondgraphs}}, but does not (yet) cover all its functions. It \emph{might} result in more appealing bond graphs, but this of course is subjective.
\end{abstract}

\begin{multicols}{2}
\tableofcontents
\end{multicols}

\section{Introduction}
    
    \subsection{Motivation}
        This package is a by-product of a project in which I was in need of a convenient way to draw bond graphs. At first, the \bondgraphs package was sufficient, but as the delivery date of the final report approached, I became less and less satisfied by the aesthetic end result of my bond graphs, especially when using multi-bonds.
        
        \begin{filecontents*}{xbg_tmp_example_bondgraphs.tex}
        \documentclass[tikz]{standalone}
        \usepackage{bondgraphs}
        \usepackage{mathpazo}
        \begin{document}
        \begin{tikzpicture}[x=15mm,y=15mm]
            \path[use as bounding box] (-25mm,-25mm) rectangle (25mm,25mm);
            \node at (-1, 0) [label=left:$ V $,bgelement] (Se) {Se};
            \node at ( 0, 0) [label=below right:$ i $,bgelement] (1j) {1};
            \node at ( 0, 1) [label=above:$ C $,bgelement] (C) {C};
            \node at ( 0,-1) [label=below:$ L $,bgelement] (L) {I};
            \node at ( 1, 0) [label=right:$ R $,bgelement] (R) {R};
            \draw (Se) edge[mbond,e_out] (1j);
            \draw (1j) edge[mbond,e_in] (C);
            \draw (1j) edge[mbond,e_out] (L);
            \draw (1j) edge[mbond,e_in] (R);
        \end{tikzpicture}
        \end{document}
        \end{filecontents*}
        \immediate\write18{pdflatex xbg_tmp_example_bondgraphs.tex}
        \begin{figure}[htpb]
            \centering
            \begin{subfigure}[t]{\linewidth}
                \centering
                \begin{tikzpicture}
                    \begin{scope}[spy using outlines={gray,magnification=8,minimum width=32mm,minimum height=22mm, connect spies,every spy on node/.append style={line width=.4pt}}]
                        \node{\includegraphics{xbg_tmp_example_bondgraphs.pdf}};
                        \spy on (0, 1.15) in node[above left]  (tipf) at (-1, 0.5);
                        \spy on (0, 0.35) in node[above right] (tail) at ( 1, 0.5);
                        \spy on (0,-1.15) in node[below right] (tipe) at ( 1,-0.5);
                    \end{scope}
                \end{tikzpicture}
                \caption{Using the \bondgraphs package.}
                \label{fig:comparisonmultibonds-bondgraphs}
            \end{subfigure}
            \begin{subfigure}[t]{\linewidth}
                \centering
                \begin{tikzpicture}
                    \begin{scope}[spy using outlines={gray,magnification=8,minimum width=32mm,minimum height=28mm, connect spies,every spy on node/.append style={line width=.4pt}}]
                        \begin{scope}[x=15mm,y=15mm]
                            \node at (-1, 0) (Se) [bge={Se}{},pin=left:$ V $];
                            \node at ( 0, 0) (1j) [bge={1}{},label=below right:$ i $];
                            \node at ( 0, 1) (C)  [bge={C}{},pin=above:$ C $];
                            \node at ( 0,-1) (L)  [bge={I}{},pin=below:$ L $];
                            \node at ( 1, 0) (R)  [bge={R}{},pin=right:$ R $];
                        \end{scope}
                        \draw[bond={multi,effort out}] (Se) -- (1j);
                        \draw[bond={multi,effort in}]  (1j) -- (C);
                        \draw[bond={multi,effort out}] (1j) -- (L);
                        \draw[bond={multi,effort in}]  (1j) -- (R);
                        \spy on (0, 1.05) in node[above left]  (tipf) at (-1, 0.5);
                        \spy on (0, 0.40) in node[above right] (tail) at ( 1, 0.5);
                        \spy on (0,-1.05) in node[below right] (tipe) at ( 1,-0.5);
                    \end{scope}
                \end{tikzpicture}
                \caption{Using the \xbondgraphs package.}
                \label{fig:comparisonmultibonds-xbondgraphs}
            \end{subfigure}
            \caption{Comparison of multi bond graph drawing.}
            \label{fig:comparisonmultibonds}
        \end{figure}
        
        \Cref{fig:comparisonmultibonds} shows an example of a simple bond graph with multi bonds, using both the \bondgraphs package and the \xbondgraphs package. The points that motivated me to rewrite the package are emphasized. Although of course subjective, most of the differences between the \bondgraphs- and the \xbondgraphs package can be argued to be improvements.
        
        Additionally, the \bondgraphs package only permits |edge|s to be bonds, making multi-segment bonds impossible. With the \refDec{bond} decoration, the \xbondgraphs package makes these multi-segment bonds possible.
        
    \subsection{Alternatives}
    
        As already mentioned, this package is based on the \bondgraphs package, but does not (yet) cover all its functions. A comparison of main package functions is shown in \cref{tab:functioncomparison}.
        
        \begin{table}[htbp]
            \centering
            \caption{Function comparison between \bondgraphs and \xbondgraphs.}
            \label{tab:functioncomparison}
            \begin{tabular}{lcc}
                \toprule
                \bfseries Function                       &        \bondgraphs         &       \xbondgraphs        \\ \midrule
                Automatic arrow barb direction           & \textcolor{green}{\cmark}  & \textcolor{green}{\cmark} \\
                Single bond drawings                     & \textcolor{green}{\cmark}  & \textcolor{green}{\cmark} \\
                In-line bonds and elements               & \textcolor{green}{\cmark}  & \textcolor{green}{\cmark} \\
                Multi bond drawings                      & \textcolor{orange}{\cmark} & \textcolor{green}{\cmark} \\
                Power (de-)mux element                   &  \textcolor{red}{\xmark}   & \textcolor{green}{\cmark} \\
                Multi-segment bonds                      &  \textcolor{red}{\xmark}   & \textcolor{green}{\cmark} \\
                Curly bond barb                          & \textcolor{green}{\cmark}  &  \textcolor{red}{\xmark}  \\
                Optional colon between element and label &  \textcolor{red}{\xmark}   & \textcolor{green}{\cmark} \\ \bottomrule
            \end{tabular}
        \end{table}
        
        A second alternative is the \textsf{bondgraph}\footnote{\url{https://ctan.org/pkg/bondgraph}} package, but because it has nearly no documentation and an incomprehensible example file, I have never tried to get it to work.
        
    \subsection{Known issues}
        
        Besides the to-dos listed in \hyperlink{sec:todolist}{the to-do list}, the following issues are known and not (yet) fixed.
        \begin{itemize}
            \item Due to using the \docAuxKey*[tikz]{double} key for multi bonds, the center of the multi bond is normally drawn in opaque white. This does not look good if the graph is drawn on a non-white background.\todo{Create a \texttt{background} color (key) that effectively changes the \texttt{double} color}
            \item 
        \end{itemize}
        Please support a starting \LaTeX nut and submit issues (and function requests) to \url{https://github.com/MaxSnippe/xbondgraphs/issues}.

\section{Basic usage}
    
    \subsection{Installation}
        This package has not yet been included in popular \LaTeX distributions, and therefore can be installed only by \dots 
        \iffalse
            downloading the installation file (\texttt{xbondgraphs.ins}) and the source (\texttt{xbondgraphs.dtx}) from \href{https://github.com/MaxSnippe/xbondgraphs}{the GitHub repository}\footnote{\url{https://github.com/MaxSnippe/xbondgraphs}} to your local TEXMF tree. It should be placed under \textvtt{\$TEXMF\$/tex/latex/local}.
        \fi
        \todo{Describe installation sequence}
        
    \subsection{Simple example}
        The following example shows an electric domain dynamic model shown as an iconic diagram, and its domain independent equivalent model shown as a bond graph. The iconic diagram icons are taken from the (undocumented) \textsf{xbondgraphs.iconix} package, that is installed alongside the \xbondgraphs package. \todo{Document the iconix package}\todo{Change iconix package into library}
\begin{dispExample*}{tikzdocexample}
\draw[line width=0.8pt] (-1,-1) 
    -- ++(0,0.5)
    -- node[voltage source,label={above:$ u $}]{} ++(0,1)
    -- node[current,label={above:$ i $}]{} ++(0,0.5)
    -- node[inductor,label={above:$ L $}]{} ++(2,0) 
    -- node[resistor2,label={above:$ R $}]{} ++(0,-2) 
    -- node[capacitor,label={above:$ C $}]{} cycle;
\begin{scope}[shift={(5,0)}]
    \node (u) at (-1.5, 0.0) [bge={Se}{},pin={left:$ u $}];
    \node (i) at ( 0.0, 0.0) [bge={1}{},label={300:$ i $}];
    \node (c) at ( 0.0, 1.5) [bge={C}{},pin={above:$ C $}];
    \node (r) at ( 1.5, 0.0) [bge={R}{},pin={right:$ R $}];
    \node (l) at ( 0.0,-1.5) [bge={L}{},pin={below:$ L $}];
    \draw[bond={effort out}] (u) -- (i);
    \draw[bond={effort in}] (i) -- (c);
    \draw[bond={effort in}] (i) -- (r);
    \draw[bond={effort out}] (i) -- (l);
\end{scope}
\end{dispExample*}
        
    \subsection{Macros}
        
        \begin{docCommand}{bge}{\oarg{xbg keys}\oarg{tikz/pgf keys}\marg{contents}}
            This macro can be used to create an in-line bond graph element. It accepts two optional arguments and one mandatory argument, firstly a list of \meta{xbg keys} as described in \cref{sec:keys}, secondly a list of \meta{tikz/pgf keys} such as color definitions, and finally a character (combination) that denotes one of the base components of bond graphs.
\begin{dispExample}
This text could say something about the \bge{MTF}-element, which is a
modulated transformer. A multidimensional specimen also exists, which is
denoted \bge[multiport]{MTF}. Extra emphasis can be given a beautiful 
color, e.g. \bge[][blue]{MTF}.
\end{dispExample}
        \end{docCommand}
        
        \begin{docCommand}{bond}{\oarg{xbg keys}\oarg{tikz/pgf keys}\marg{path}}
            This macro can be used to create an in-line bond. It accepts the same optional arguments as the \refCom{bge} macro, and the final argument is a \Tikz compatible path description.
\begin{dispExample}
This text could use an in-line \bond{(0,0)--(1,0)} just for clarification.
If the bond is multidimensional, lets draw it as such \bond[multi]{(0,0)
--(1,0)}. And to give additional emphasis, lets give it a beautiful color
\bond[multi][blue]{(0,0)--(1,0)}.
\end{dispExample}
        \end{docCommand}
        
    \subsection{\Tikz styles}
        \label{sec:tikzstyles}
        
        \begin{docKey}[tikz]{bond}{=\meta{options}}{default |empty|}
            When the \refKey{/tikz/bond} style is provided to a \Tikz drawing command (e.g. \cs{draw}, \cs{path}), the \refDec{bond} decoration is applied. Additionally, some options can be given as argument. Note that this style has to be given to every path that must be a bond, because it uses a decoration. Alternatively, the key \docAuxKey*[tikz]{decorate} must be given.
\begin{dispExample*}{tikzdocexample}
\draw[bond] (0,0) -- ++(1,0);
\begin{scope}[bond={multi}]
    \draw[decorate] (1.5,0) -- ++(1,0);
    \draw (3,0) -- ++(1,0);
\end{scope}
\end{dispExample*}
        \end{docKey}
        \begin{docKey}[tikz]{bond graph element}{=\marg{contents}\marg{options}}{default \brackets{empty}\brackets{empty}}
            This style can be used for \Tikz nodes. It takes two arguments, the first being the node contents, and the second a list of \meta{key}|=|\meta{value} options. The \meta{contents} will be printed in \cs{mathbf} for a normal element, and in \cs{mathbb} for a multiport element. Note that most fonts do not have glyphs for lowercase blackboard bold.
\begin{dispExample*}{tikzdocexample}
\node at (0,0) [bond graph element={I}{}];
\node at (1,0) [bond graph element={C}{multiport,n=3}];
\node at (2.5,0) [bond graph element={motor}{word}];
\node at (4.5,0) [bond graph element={BODY}{word,multiport,n=6}];
\end{dispExample*}
        \end{docKey}
        \begin{docKey}[tikz]{bge}{=\marg{options}\marg{contents}}{default \brackets{|empty|}\brackets{|empty|}}
            This key is provided as a shorter alias for \refKey{/tikz/bond graph element}.
        \end{docKey}
        \begin{docKey}[tikz]{effort}{=\meta{contents}}{default |empty|}
            This key only makes sense if given to a |to| or |edge| path command. It places a edge node (following convention) \emph{above} the middle of the path segment with \meta{contents} to describe the effort variable of the bond.
        \end{docKey}
        \begin{docKey}[tikz]{flow}{=\meta{contents}}{default |empty|}
            This key is the counterpart of \refKey{/tikz/effort}, and places a edge node \emph{below} the middle of the path segment to describe the flow variable of the bond.
        \end{docKey}
\begin{dispExample*}{tikzdocexample}
\foreach \ang in {0,60,...,359}{
    \draw[bond] (0,0) to[flow=$ f $] ++(\ang:1.5);
    \draw[bond] (3.5,0) to[effort=$ e $] ++(\ang:1.5);
}
\end{dispExample*}
        \begin{docKey}[tikz]{mux}{=\meta{options}}{default |empty|}
            This style applies the \refShp{mux} shape, and sets the minimum size of the node according to the number of in- and outputs, and the in-/output spacing. The node contents are set to \brackets{}, and the fill of the shape to black. The \refShp{mux} shape defines an anchor |input |\meta{n} for every input and |output |\meta{n} for every output.
            
            Inputs are always placed on the left hand side of the shape, and outputs on the right hand side. The shape can of course be rotated with the \docAuxKey*[tikz]{rotate} key.
\begin{dispExample*}{tikzdocexample}
\node (mux) [mux={mux inputs=2,mux outputs=3}];
\foreach \i in {1,2}{\draw[<-] (mux.input \i) -- ++(-1,0);}
\foreach \i in {1,...,3}{\draw[->] (mux.output \i) -- ++(1,0);}
\node at (3,0) (mux2) [mux={mux inputs=2,mux outputs=3},rotate=180];
\foreach \i in {1,2}{\draw[<-] (mux2.input \i) -- ++(1,0);}
\foreach \i in {1,...,3}{\draw[->] (mux2.output \i) -- ++(-1,0);}
\end{dispExample*}
        \end{docKey}
        
\section{Keys}
    \label{sec:keys}
    
    \begin{docCommand}{xbgset}{\brackets{\meta{key-value list}}}
        The \refCom{xbgset} macro can be used to set the keys described in this section. This macro basically only changes the current \pgfkeyspkg directory to \docAuxKey*{/xbg} and sets the keys presented in \meta{key-value list}.
    \end{docCommand}
    
    \subsection{Bond behavior}
        
        \begin{xbgkey}{single bond width}{=\meta{dimension}}{no default, initially |0.8pt|}
            This key sets the width of the visible bond lines, including the two lines of the multi bonds, and the causality stroke.
\begin{dispExample*}{tikzdocexample}
\foreach [count = \i from 0] \lw in {0.1pt, 0.4pt, 0.8pt, 1pt, 2pt, 4pt}{
    \begin{scope}[/xbg/single bond width=\lw]
        \draw[bond] (1.5*\i,0) -- node[above]{\lw} ++(1,0);
        \draw[bond={multi}] (1.5*\i,-0.5) -- ++ (1,0);
    \end{scope}
}
\end{dispExample*}
            As shown in the example, the multi bond width does not scale with the single bond width.\todo{Maybe this should be added, for simple and consistent scaling.}
        \end{xbgkey}
        
        \begin{xbgkey}{multi bond width}{=\meta{dimension}}{no default, initially |3.2pt|}
            This key sets the total width of the multi bonds, effectively determining the width of the gap between the two drawn lines, in combination with the former option ($ w_{\mathrm{multi}} = 2w_{\mathrm{single}} + w_{\mathrm{gap}} $). Additionally, this width is used to scale the causality stroke's width ($ w_{\mathrm{cstroke}}=2w_{\mathrm{multi}} $ for single bonds, and $ w_{\mathrm{cstroke}}=3w_{\mathrm{multi}} $ for multi bonds) and the length of the barb is scaled such that it ends parallel with the causality stroke.\todo{The causal stroke scaling could be a separate option, maybe later.}
\begin{dispExample*}{tikzdocexample}
\foreach [count = \i from 0] \lw in {2pt, 3.2pt, 4pt, 5pt, 8pt}{
    \begin{scope}[/xbg/multi bond width=\lw]
        \draw[bond] (1.5*\i,0) -- node[above]{\lw} ++ (1,0);
        \draw[bond={multi}] (1.5*\i,-0.5) -- ++ (1,0);
    \end{scope}
}
\end{dispExample*}
            The default ratio between single- and multi bond widths (1:4) ought to be a good rule of thumb when scaling the line widths.
        \end{xbgkey}
                    
        \begin{xbgkey}{bond label color}{=\meta{color}}{no default, initially |green!50!black| [\colorpreview{green!50!black}]}
            This key can be used to set the color of all effort or flow nodes placed along the bonds. The key sets the color \docColor{BondLabelColor} using |\colorlet| to the color described with \meta{color}. 
        \end{xbgkey}
        
        \begin{xbgkey}{error color}{=\meta{color}}{no default, initially |red| [\colorpreview{red}]}
            This key sets the color for erroneous causality strokes. The color is named \docColor{Error} and is set to the value of \meta{color} using |\colorlet|.
        \end{xbgkey}
        
        \begin{xbgkey}{differential color}{=\meta{color}}{no default, initially |orange| [\colorpreview{orange}]}
            Similarly to \refKey{/xbg/error color}, this key sets the color \docColor{Differential} to the value of \meta{color}. This can be used for causality strokes with (non-preferred) differential causality.
        \end{xbgkey}
        
        \begin{xbgkey}{gray}{}{initially unset}
            This key is a shortcut to set the following colors:
            
            \begin{tabular}{llc}
                \docColor{BondLabelColor}    & |gray|           &      \colorpreview{gray}      \\
                \docColor{ElementLabelColor} & |gray|           &      \colorpreview{gray}      \\
                \docColor{Differential}      & |white!60!black| & \colorpreview{white!60!black} \\
                \docColor{Error}             & |white!30!black| & \colorpreview{white!30!black}
            \end{tabular}
            
            For if you want to limit the number of colors in your document, for whatever reason.
        \end{xbgkey}
        
        \begin{xbgkey}{barb direction rule}{=\meta{direction}}{no default, initially |always below|}
            This key can be used to set the automated direction of the arrow barb. Accepted values for \meta{direction} are:
            \begin{itemize}
                \item \docValue{always below}
                \item \docValue{left below}
            \end{itemize}
            See the following example for how the arrow barb behaves.
\begin{dispExample*}{tikzdocexample}
\foreach [count=\i from 0] \opt in {left below,always below}{
    \begin{scope}[x=1.25cm,y=1.25cm]
        \xbgset{barb direction rule=\opt}
        \foreach \a in {0,15,...,359}{
            \draw[bond] (2*\i,0) -- ++(\a:1);
        }
        \node at (2*\i,-1.5) [anchor=base,font=\ttfamily]{\opt};
    \end{scope}
}
\end{dispExample*}
        \end{xbgkey}
        
        \begin{xbgkey}{barb angle}{=\meta{angle}}{no default, initially |40|}
            This key sets the barb angle of the bond's barb arrow tip. It can be set in any desired (pgf-)scope, including globally. The accepted values are limited between 15 and 85 degrees, to prevent distorted arrow tips.
\begin{dispExample*}{tikzdocexample}
\foreach [count=\i] \ang in {15,25,35,40,45,65,85}{
    \begin{scope}[shift={(\i*1.2,0)}]
        \draw[bond={barb angle=\ang}](0,0) --node[above]{\ang} ++(1,0);
    \end{scope}
}
\end{dispExample*}
        \end{xbgkey}
        
        \begin{xbgkey}{effort out}{=\meta{options}}{no default, initially unset}
            This key places the causality stroke of a bond at the end (where the barb is placed). The optional argument is passed on to the \docAuxKey[xbg]{causal stroke} style which is used for the \textbar-arrow tip. For differential or erroneous causality, the colors \docColor{Differential} or \docColor{Error} can be passed to the causality option. The default colors are red (\colorpreview{red}) for \docColor{Error} and orange (\colorpreview{orange}) for \docColor{Differential}.
        \end{xbgkey}
        
        \begin{xbgkey}{flow in}{=\meta{options}}{no default, initially unset}
            This key is an alias for the \refKey{/xbg/effort out} key.
        \end{xbgkey}
        
        \begin{xbgkey}{effort in}{=\meta{options}}{no default, initially unset}
            This key is the counterpart of the \refKey{/xbg/effort out} key. It places the causality stroke at the beginning of the bond.
        \end{xbgkey}
        
        \begin{xbgkey}{flow out}{=\meta{options}}{no default, initially unset}
            This key is an alias for the \refKey{/xbg/effort in} key. The causality cannot be set to both flow out and effort out, these keys are mutually exclusive, and the last given key is decisive.
        \end{xbgkey}
        
        \begin{xbgkey}{unset causality}{}{initially set}
            If one of the former causality keys is set, this key un-sets them all (both). This might prove useful when one of the former is set in a larger scope.
        \end{xbgkey}
\begin{dispExample*}{tikzdocexample}
\draw[bond={effort out={Error}}] (0,0) -- ++(1,0);
\draw[bond={flow in={width=1.5em,Differential}}] (1.5,0) -- ++(1,0);
\draw[bond={effort in={blue,line width=4pt}}] (3,0) -- ++(1,0);
\draw[bond={flow out},green] (4.5,0) to[effort={$ e $}] ++(1,0);
\draw[bond={flow out, unset causality}] (6,0) to[flow={$ f $}] ++(1,0);
\end{dispExample*}

        \begin{xbgkey}{multi}{}{initially unset}
            When this key is called, the bond is changed to a multi bond, with a different arrow head, and double lines. All other options are still applicable, e.g. causality options.
\begin{dispExample*}{tikzdocexample}
\draw[bond={multi,effort out={Differential}}] (0,0) -- ++(1,0);
\draw[bond={multi},red] (1.5,0) -- ++(1,0);
\draw[bond={multi,flow in}] (3,0) to[effort={$ \mathbf{V}_{abc} $}] ++(1,0);
\end{dispExample*}
        \end{xbgkey}
        
    \subsection{Element behavior}
        
        \begin{xbgkey}{element label color}{=\meta{color}}{no default, initially |blue| [\colorpreview{blue}]}
            This key can be used to set the color of the labels (and pins) of bond graph elements. The key sets the color \docColor{ElementLabelColor} using |\colorlet| to the color described with \meta{color}. 
        \end{xbgkey}
        
        \begin{xbgkey}{multiport}{=true\textbar false}{default |true|, initially |false|}
            If the element depicts a multidimensional element, this might be emphasized with the |multiport| option. This will print the node contents in blackboard bold math font. Generally this should be combined with the next option (|n|), unless all elements in the graph have the same order.
\begin{dispExample*}{tikzdocexample}
\node (mtf) [bge={MTF}{multiport}];
\draw[bond={multi,effort in}] (-2,0) -- (mtf);
\draw[bond={multi,effort in}] (mtf) -- (2,0);
\end{dispExample*}
            As mentioned before, most fonts do not have glyphs for lower case blackboard bold. For elements as \bge{MSe}s or \bge{MSf}s, this might be a problem, but this could be solved by not explicitly setting the \refKey{/xbg/multiport} key, but using blackboard bold in the \meta{contents} field. This can be solved with some packages, e.g. \textsf{bbm} or \textsf{mathbbol}, but I will leave that for you as end user to solve\footnote{Personally, I do not like these solutions.}.\todo{The multiport option should ignore lower case symbols and just print them with normalfont}
\begin{dispExample*}{tikzdocexample}
\node [bge={MSe}{multiport}];
\node at (1,0) [bge={\mathbb{MS}e}{}];
\node at (2,0) [bge={MSf}{multiport}];
\node at (3,0) [bge={\mathbb{MS}f}{}];
\end{dispExample*}
        \end{xbgkey}
        
        \begin{xbgkey}{n}{=\meta{order}}{no default, initially |1|}
            The order of the bond graph element can be given with the |n=|\marg{order} option. This will be added to the node contents as subscript, if unequal to the default (1).
\begin{dispExample*}{tikzdocexample}
\node (I) [bge={I}{multiport,n=6}];
\draw[bond={multi,effort out}] (-2,0) -- (I);
\end{dispExample*}
        \end{xbgkey}
        
        \begin{xbgkey}{word}{=true\textbar false}{default |true|, initially |false|}
            Following convention, word bond graph elements are depicted as drawn ellipsoids with a descriptive name. This can be achieved with this key.
\begin{dispExample*}{tikzdocexample}
\node (motor) [bge={Motor}{word}];
\node (wheel) [right=of motor,bge={Wheel}{word}];
\node [right=of wheel,bge={RIGIDBODY}{multiport,n=6,word}];
\end{dispExample*}
        \end{xbgkey}
        
        \begin{xbgkey}{label colon}{}{default |true|, initially |false|}
            Common practice when drawing bond graphs is using a colon between a (non-junction) element and its label. This can be achieved in two ways when using this package.
\begin{dispExample*}{tikzdocexample}
\node [bge={C}{},label=north:$ \frac{1}{k} $];
\node at (1,0) [bge={C}{},pin=north:$ \frac{1}{k} $];
\node at (2,0) [bge={C}{label colon},label=north:$ \frac{1}{k} $];
\end{dispExample*}
            The |pin| method is shorter, but the \refKey{/xbg/label colon} key can also be set in a scope.
        \end{xbgkey}
        
    \subsection{Mux behavior}
        
        The following keys are used when drawing the \refShp{mux} shape. They can be given to the \refKey{/tikz/mux} style as arguments.
        
        \begin{xbgkey}{mux inputs}{=\meta{integer}}{no default, initially |1|}
            This key sets the number of inputs of the \refShp{mux} element.
        \end{xbgkey}
        
        \begin{xbgkey}{mux outputs}{=\meta{integer}}{no default, initially |1|}
            This key sets the number of outputs of the \refShp{mux} element.
        \end{xbgkey}
        
        \begin{xbgkey}{mux io spacing}{=\meta{dimension}}{no default, initially |5mm|}
            This key sets the distance between two consecutive in- or outputs. Together with the maximum number of in-/outputs, it also determines the height of the \refShp{mux} element.
        \end{xbgkey}
        
\section{Miscellaneous \pgf definitions}
        
    \begin{docDecoration}{bond}{}
        This decoration is used to determine the direction of the arrow tip barb. It uses the \docAuxCommand*{pgfdecoratedangle} in a |case| statement. Note that \docAuxCommand*{pgfdecoratedangle} holds the angle of the \emph{start} of the current path segment, so if the last segment of a path is curved, it might lead to a incorrect direction for the barb. 
    \end{docDecoration}
    
    \begin{docShape}{mux}{}
    
    \end{docShape}
    
    \begin{docArrowTip}{Single Bond Barb}{}
    
    \end{docArrowTip}
    
    \begin{tikzpicture}
    
    % Calculate actual path points
    \pgfmathsetlengthmacro{\tipx}{\singlebondwidth}
    \pgfmathsetlengthmacro{\tipy}{0pt}
    \pgfmathsetlengthmacro{\backx}{-1/tan(\barbangle)*(\multibondwidth-0.5*cos(\barbangle)*\singlebondwidth) + \singlebondwidth}
    \pgfmathsetlengthmacro{\backy}{\multibondwidth - 0.5*cos(\barbangle)*\singlebondwidth}
    % Calculate points of outer dimensions as needed by pgf
    \pgfmathsetlengthmacro{\hullpointx}{\backx + 0.5*\singlebondwidth*sin(\barbangle)}
    \pgfmathsetlengthmacro{\hullpointy}{\multibondwidth}
    \pgfmathsetlengthmacro{\tipendx}{0.5*\singlebondwidth/tan(\barbangle/2) + \tipx}
    \pgfmathsetlengthmacro{\tipendy}{-0.5*\singlebondwidth}
    
    \begin{scope}[line width = \singlebondwidth,/xbg/multi bond width=\multibondwidth]
    % Existing line
    \draw (0,0) -- (-5,0);
    
    % Arrow tip
    \draw[arrow tip demonstration,-{Single Bond Barb[left].|[width=2*\multibondwidth]}] (0,0) -- (\tipendx+\singlebondwidth,\tipy);
    %					\draw[arrow tip demonstration] (0,0) -- (\tipx,\tipy) -- (\backx,\backy);
    %					\draw[arrow tip demonstration] (\tipendx+0.5*\singlebondwidth,-\multibondwidth) -- (\tipendx+0.5*\singlebondwidth,\multibondwidth);
    \draw[arrow tip line style] (0,0) -- (\tipx,\tipy) -- (\backx,\backy);
    \end{scope}
    
    \begin{scope}[annotation style]
    \begin{scope}[xshift=\tipendx+2.5mm]
    \draw (0,\tipendy) -- node[right]{$ y_a $} (0,\tipy);
    \draw (0,\tipy) -- node[right]{$ y_b $} (0,\backy);
    \draw (0,\backy) -- node[right]{$ y_c $} (0,\hullpointy);
    \end{scope}
    \begin{scope}[yshift=\hullpointy+2.5mm]
    \draw (\backx,0) -- node[above]{$ x_a $} (\hullpointx,0);
    \draw (\hullpointx,0) -- node[above]{$ x_b $} (0,0);
    \draw (0,0) -- node[above]{$ x_c $} (\tipx,0);
    \draw (\tipx,0) -- node[above]{$ x_d $} (\tipendx,0);
    \end{scope}
    \end{scope}
    
    \begin{scope}[help lines]
    \draw (\backx,\hullpointy) -- (\hullpointx,\hullpointy-|\tipendx+3mm,0);
    \draw (\tipx,\tipy) -- (\tipx,\tipy-|\tipendx+3mm,0);
    \end{scope}
    
    \end{tikzpicture}
    
    \begin{docArrowTip}{Multi Bond Barb}{}
    
    \end{docArrowTip}
    
    \begin{tikzpicture}
    
    % Calculate actual path points
    \pgfmathsetlengthmacro{\startx}{0pt}
    \pgfmathsetlengthmacro{\starty}{-0.5*\multibondwidth+0.5*\singlebondwidth}
    \pgfmathsetlengthmacro{\tipx}{(\multibondwidth-\singlebondwidth)/tan(\barbangle)}
    \pgfmathsetlengthmacro{\tipy}{-0.5*\multibondwidth + 0.5*\singlebondwidth}
    \pgfmathsetlengthmacro{\backy}{1.5*\multibondwidth - 0.5*\singlebondwidth*cos(\barbangle)}
    \pgfmathsetlengthmacro{\backx}{-(\backy+\tipy)/tan(\barbangle)}
    % Calculate points of outer dimensions as needed by pgf
    \pgfmathsetlengthmacro{\hullpointx}{\backx + 0.5*\singlebondwidth*sin(\barbangle)}
    \pgfmathsetlengthmacro{\hullpointy}{1.5*\multibondwidth}
    \pgfmathsetlengthmacro{\tipendx}{0.5*\singlebondwidth/tan(\barbangle/2) + \tipx}
    \pgfmathsetlengthmacro{\tipendy}{-0.5*\multibondwidth}
    
    \begin{scope}[line width = \singlebondwidth]
    % Existing line
    \draw (0,\starty) -- (-5,\starty); 
    \draw (0,-\starty) -- (-5,-\starty); 
    
    % Arrow tip
%    \draw[double,double distance={\multibondwidth-2*\singlebondwidth},arrow tip demonstration,-{Multi Bond Barb[left].|[width=3*\multibondwidth]}] (0,0) -- (\tipendx+\singlebondwidth,0);
    \draw[arrow tip demonstration] (\startx,\starty) -- (\tipx,\tipy) -- (\backx,\backy);
    \draw[arrow tip line style] (\startx,\starty) -- (\tipx,\tipy) -- (\backx,\backy);
    \draw[arrow tip line style,green] (\tipx,\tipy) -- ++ (180-\barbangle:10cm);
    \end{scope}
    
    \node[draw,cross out] at (\hullpointx,\hullpointy){};
    
    \begin{scope}[annotation style]
    \draw[<->] (\tipx+5mm,\tipy) -- node[right]{$ 2a $} ++ (0,2*\multibondwidth);
    \draw[<->] (\startx,\starty-5mm) -- node[below]{} (\tipx,\tipy-5mm);
    \end{scope}
    
    \end{tikzpicture}
    
\section{Examples}
	
	\begin{tikzpicture}
		\node (Vabc) at (0,0) [bond graph element={\mathbb{MS}e}{n=3},label={west:$ V_{abc} $}];
		\node (iabc) [right=of Vabc,bond graph element={1}{},label={300:$ i_{abc} $}];
		\node (Rs) [above=of iabc,bond graph element={R}{},pin={north:$ R_s $}];
		\node (Ls) [below=of iabc,bond graph element={L}{},pin={south:$ L_s $}];
		\node (Klambda) [right=of iabc,bond graph element={MGY}{},pin={south:$ K\Lambda(\theta_e) $}];
		\node (mux1) [right=of Klambda,mux={mux outputs=3}];
		\node (omegae) [right=of mux1,bond graph element={1}{},label={300:$ \omega_e $}];
		\node (1overp) [right=of omegae,bond graph element={TF}{},pin={south:$ \frac{1}{p} $}];
		\node (omegam) [right=of 1overp,bond graph element={1}{},label={300:$ \omega_m $}];
		\node (Dr) [above=of omegam,bond graph element={R}{},pin={north:$ D_r $}];
		\node (Jr) [below=of omegam,bond graph element={I}{},pin={south:$ J_r $}];
		
		\begin{scope}[bond={multi}]
			\draw[bond={effort out}] (Vabc) -- (iabc);
			\draw[bond={effort in}] (iabc) -- (Rs);
			\draw[bond={effort out}] (iabc) -- (Ls);
			\draw[bond={effort in}] (iabc) -- (Klambda);
			\draw[bond={effort out}] (Klambda) -- (mux1.input 1);
		\end{scope}
		
		\foreach \i in {1,2,3}{
			\draw[bond={effort out}] (mux1.output \i) -- ++ (0.8,0) -- (omegae);
		}
		\draw[bond={effort out}] (omegae) -- (1overp);
		\draw[bond={effort out}] (1overp) -- (omegam);
		\draw[bond={effort out}] (omegam) -- (Jr);
		\draw[bond={effort in}] (omegam) -- (Dr);
	\end{tikzpicture}

	\begin{tikzpicture}[
		output/.style={draw=blue,fill=blue,pos=0,right},
		input/.style={draw=red,fill=red,pos=1,left},
		every node/.style={text=white,font=\tiny,inner sep=0.5pt}]
		\foreach \io [count=\i from 0,count=\j from 1,remember=\io as \lastio (initially 1)] in {4,3,8,5,1}{
			\node at (\i+1,0) (mux\j) [mux={mux inputs=\lastio,mux outputs=\io}];
			\foreach \k in {1,...,\lastio}{
				\ifnum\i=0\relax
					\draw (0,0) -- (mux\j.input \k) node[input]{I\k};
				\else
					\draw (mux\i.output \k) -- (mux\j.input \k) node[output]{O\k} node[input]{I\k};
				\fi
			}
		}
		\draw (mux5.output 1) -- ++(1,0) node[output]{O1};
	\end{tikzpicture}
 