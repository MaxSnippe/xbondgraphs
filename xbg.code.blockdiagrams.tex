%%
%% This is file `xbg.code.blockdiagrams.tex',
%% generated with the docstrip utility.
%%
%% The original source files were:
%%
%% xbondgraphs.dtx  (with options: `xbg-code-blockdiagrams')
%% 
%% This is a generated file.
%% 
%% Copyright (C) 2018 by M.J.W. Snippe
%% 
%% This work may be distributed and/or modified under the
%% conditions of the LaTeX Project Public License, either version 1.3
%% of this license or (at your option) any later version.
%% The latest version of this license is in
%% 
%%     http://www.latex-project.org/lppl.txt
%% 
%% and version 1.3 or later is part of all distributions of LaTeX
%% version 2005/12/01 or later.
%% 
%% This work has the LPPL maintenance status `maintained'.
%% 
%% The Current Maintainer of this work is M.J.W. Snippe
%% 
%% This work consists of the files found at https://github.com/MaxSnippe/xbondgraphs.
%% 
\makeatletter
\newdimen\xbg@bd@block@iospacing
\newdimen\xbg@bd@block@height

\newif\ifxbg@bd@block@flip
\newcounter{xbg@sum@elements}

\tikzset{
    sum/.style={
        append after command={
            \pgfextra{
                \setcounter{xbg@sum@elements}{0}%
                \foreach \j in {#1}{\stepcounter{xbg@sum@elements}}
                \pgfmathsetmacro\@noelem{max(4,\thexbg@sum@elements)}
                \def\@plus{+}%
                \def\@min{-}%
                \foreach[count=\i from 0] \@el in {#1}{
                    \edef\ex@el{\@el}
                    \pgfmathsetmacro\@ang{360/\@noelem*\i + 180}%
                    \draw (\tikzlastnode) ++(\@ang:0.25) ++(\@ang+90:0.15)%
                        \ifx\ex@el\@plus%
                                +(-0.5mm,0mm) -- +(0.5mm,0mm)
                                +(0mm,-0.5mm) -- +(0mm,0.5mm)
                        \else
                            \ifx\ex@el\@min%
                                +(-0.5mm,0mm) -- +(0.5mm,0mm)
                            \fi
                        \fi;
                }
            }
        },
        node contents={},
        circle,
        draw,
        minimum size=4mm,
    },
    sum/.default = {+,-},
    signalname/.style={},
}

\tikzset{
    block/.code={
        \xbgset{
            block/.cd,
            #1
        }
        \pgfmathsetlengthmacro{\xbg@bd@block@height}{%
            max(10mm, int(max(\xbg@bd@block@inputs,\xbg@bd@block@outputs) * \xbg@bd@block@iospacing))}
        \xbgset{block/block template}
    },
    signalname/.code={
        \tikzset{
            align=center,
        }
    },
    splitter/.style args ={
        fill,
        inner sep = 0pt,
        minimum size = 1.5mm,
        circle,
    },
}
\xbgset{
    block/.is family,
    block/.cd,
        inputs/.code={
            \pgfmathparse{int(#1)}
            \let\xbg@bd@block@inputs=\pgfmathresult
        },
        inputs=1,
        outputs/.code={
            \pgfmathparse{int(#1)}
            \let\xbg@bd@block@outputs=\pgfmathresult
        },
        outputs=1,
        io spacing/.code=\setlength\xbg@bd@block@iospacing{#1},
        io spacing=5mm,
        flip/.is choice,
        flip/true/.code={\xbg@bd@block@fliptrue\def\xbg@bd@block@flipbool{1}},
        flip/false/.code={\xbg@bd@block@flipfalse\def\xbg@bd@block@flipbool{0}},
        flip/.default=true,
        flip=false,
        block template/.code=\tikzset{
            draw,
            align=center,
            minimum width = 15mm,
            minimum height = \xbg@bd@block@height,
            shape=block,
            text = Iconic Contents,
            prefix after command= {
                \pgfextra{
                    \tikzset{
                        every label/.style={
                            Iconic Label,
                        }
                    }
                }
            },
        },
}
\pgfdeclareshape{block}{
    \inheritsavedanchors[from={rectangle}]
    \savedanchor\centerpoint{%
        \pgf@x=.5\wd\pgfnodeparttextbox%
        \pgf@y=.5\ht\pgfnodeparttextbox%
        \advance\pgf@y by-.5\dp\pgfnodeparttextbox%
    }
    \inheritsavedanchors[from=rectangle]
    \inheritanchorborder[from=rectangle]
    \inheritanchor[from=rectangle]{north}
    \inheritanchor[from=rectangle]{north west}
    \inheritanchor[from=rectangle]{north east}
    \inheritanchor[from=rectangle]{center}
    \inheritanchor[from=rectangle]{text}
    \inheritanchor[from=rectangle]{west}
    \inheritanchor[from=rectangle]{east}
    \inheritanchor[from=rectangle]{mid}
    \inheritanchor[from=rectangle]{mid west}
    \inheritanchor[from=rectangle]{mid east}
    \inheritanchor[from=rectangle]{base}
    \inheritanchor[from=rectangle]{base west}
    \inheritanchor[from=rectangle]{base east}
    \inheritanchor[from=rectangle]{south}
    \inheritanchor[from=rectangle]{south west}
    \inheritanchor[from=rectangle]{south east}
    \savedmacro\blockinputs{%
        \pgfmathparse{int(\xbg@bd@block@inputs)}%
        \let\blockinputs=\pgfmathresult}%
    \savedmacro\blockoutputs{%
        \pgfmathparse{int(\xbg@bd@block@outputs)}%
        \let\blockoutputs=\pgfmathresult}%
    \savedmacro\blockmaxio{%
        \pgfmathparse{int(max(\xbg@bd@block@inputs,\xbg@bd@block@outputs))}%
        \let\blockmaxio=\pgfmathresult}%
    \savedmacro\blockflip{%
        \pgfmathparse{\xbg@bd@block@flipbool}%
        \let\blockflip=\pgfmathresult}%
    \saveddimen\halfwidth{\pgfmathsetlength\pgf@x{%
        \pgfkeysvalueof{/pgf/minimum width}/2}\pgfmathresult}
    \saveddimen\halfheight{\pgfmathsetlength\pgf@x{%
        \pgfkeysvalueof{/pgf/minimum height}/2}\pgfmathresult}
    \saveddimen\iospacing{\pgfmathsetlength\pgf@x{%
        \xbg@bd@block@iospacing}\pgfmathresult}
    \inheritbackgroundpath[from={rectangle}]
    \pgfutil@g@addto@macro\pgf@sh@s@block{%
        \pgfmathloop%
        \ifnum\pgfmathcounter>\blockinputs\relax%
        \else%
            \pgfutil@ifundefined{pgf@anchor@block@input \pgfmathcounter}{%
                \expandafter\xdef\csname pgf@anchor@block@input %
                    \pgfmathcounter\endcsname{\noexpand%
                    \pgf@sh@lib@block@in@anchor{\pgfmathcounter}%
                }%
            }{}
        \repeatpgfmathloop%
        \pgfmathloop%
        \ifnum\pgfmathcounter>\blockoutputs\relax%
        \else%
            \pgfutil@ifundefined{pgf@anchor@block@output \pgfmathcounter}{%
                \expandafter\xdef\csname pgf@anchor@block@output %
                \pgfmathcounter\endcsname{\noexpand%
                    \pgf@sh@lib@block@out@anchor{\pgfmathcounter}%
                }%
            }{}
        \repeatpgfmathloop%
    }%
}

\def\pgf@sh@lib@block@in@anchor#1{%
    \pgf@process{\centerpoint}%
    \pgf@ya=\pgf@y%
    \ifnum\blockflip=0\relax%
        \pgf@process{\southwest}%
    \else%
        \pgf@process{\northeast}%
    \fi%
    \pgfmathsetlength\pgf@y{\pgf@ya + (0.5*(\blockinputs+1)-#1)*\iospacing}
}
\def\pgf@sh@lib@block@out@anchor#1{%
    \pgf@process{\centerpoint}%
    \pgf@ya=\pgf@y%
    \ifnum\blockflip=0\relax%
        \pgf@process{\northeast}%
    \else%
        \pgf@process{\southwest}%
    \fi%
    \pgfmathsetlength\pgf@y{\pgf@ya + (0.5*(\blockoutputs+1)-#1)*\iospacing}
}
\makeatother
\endinput
%%
%% End of file `xbg.code.blockdiagrams.tex'.
