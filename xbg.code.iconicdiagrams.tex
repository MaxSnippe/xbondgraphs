%%
%% This is file `xbg.code.iconicdiagrams.tex',
%% generated with the docstrip utility.
%%
%% The original source files were:
%%
%% xbondgraphs.dtx  (with options: `xbg-code-iconicdiagrams')
%% 
%% This is a generated file.
%% 
%% Copyright (C) 2018 by M.J.W. Snippe
%% 
%% This work may be distributed and/or modified under the
%% conditions of the LaTeX Project Public License, either version 1.3
%% of this license or (at your option) any later version.
%% The latest version of this license is in
%% 
%%     http://www.latex-project.org/lppl.txt
%% 
%% and version 1.3 or later is part of all distributions of LaTeX
%% version 2005/12/01 or later.
%% 
%% This work has the LPPL maintenance status `maintained'.
%% 
%% The Current Maintainer of this work is M.J.W. Snippe
%% 
%% This work consists of the files found at https://github.com/MaxSnippe/xbondgraphs.
%% 
\makeatletter
\usetikzlibrary{decorations.pathreplacing,decorations.pathmorphing,patterns}

\xbgset{,
    iconic label color/.code=\colorlet{Iconic Label}{#1},
    iconic label color=blue,
    iconic contents color/.code=\colorlet{Iconic Contents}{#1},
    iconic contents color=blue,
}
\tikzset{
signal/.style args = {#1}{
-{Latex[length=6pt,width=6pt]},
thick,
#1,
},
signal/.default = {},
msignal/.style args = {#1}{
double,
double distance = 2.5pt,
-{Latex[length=7pt,width=7pt]},
thick,
#1,
},
msignal/.default = {},
}
\tikzset{
iconic block/.style args ={
draw,
thick,
minimum height = 1cm,
minimum width = 1cm,
},
iconic block/.default = {},
}

\tikzset{
sinewavegen/.style args = {#1}{
iconic block,
path picture = {
\begin{scope}[x=1mm,y=2.5mm]
\draw (-4mm,3.5mm) -- (-4mm,-3.5mm);
\draw (-4.5mm,0mm) -- (4.5mm,0mm);
\draw[Iconic Contents,thick] (-4mm,0) sin ++(1,1) cos ++(1,-1) sin ++(1,-1) cos ++(1,1) sin ++(1,1) cos ++(1,-1) sin ++(1,-1) cos ++(1,1);
\end{scope}
}
},
motionprofilegen/.style args = {#1}{
iconic block,
path picture = {
\draw (-4mm,3.5mm) -- (-4mm,-3.5mm);
\draw (-4.5mm,-3mm) -- (4.5mm,-3mm);
\begin{scope}[Iconic Contents,thick,smooth,variable=\t,x=1cm,y=1cm]
\pgfmathsetmacro{\hm}{0.5}
\pgfmathsetmacro{\tm}{0.7}
\pgfmathsetmacro{\ts}{-0.4}
\pgfmathsetmacro{\ofs}{-0.3}
\draw[domain=\ts:\ts+\tm/4] plot ({\t},{\ofs + 16/3*\hm*((\t-\ts)/\tm)^3});
\draw[domain=\ts+\tm/4:\ts+3*\tm/4] plot ({\t},{\ofs + 32*\hm/(\tm)^3*(\tm*(\t-\ts)^2/4 - (\t-\ts)^3/6 - (\tm)^2*(\t-\ts)/16 + (\tm)^3/192)});
\draw[domain= \ts+3*\tm/4:\ts+\tm]  plot ({\t},{\ofs + 32*\hm/(\tm)^3*(-\tm*(\t-\ts)^2/2+(\t-\ts)^3/6+(\tm)^2*(\t-\ts)/2-13*(\tm)^3/96)});
\draw[domain=\ts+\tm:\ts+\tm+0.1] plot ({\t},{\ofs+\hm});
\end{scope}
}
}
}

\tikzset{
integrator/.style args = {#1}{
iconic block,
path picture = {
\node[Iconic Contents]{\ensuremath{\int}};
}
}
}
\tikzset{
component/.style={
inner sep = 0pt,
prefix after command= {
\pgfextra{
\tikzset{
every label/.style={
text=Iconic Label,
label distance = 0pt,
}
}
}
},
},
node component/.style={
component
},
line component/.style={
component,
sloped,
allow upside down,
pos = .5,
},
hydrline/.style={
draw,
line width = 1pt,
double,
double distance between line centers=3mm,
},
flowsource/.style 2 args={
component,
minimum height = 1cm,
minimum width = 1.5cm,
label={#2:#1},
path picture={%
\draw[hydrline] (-.5,0)--(.5,0);
\draw[fill=Background] (0,.15) circle (.15);
\draw[fill=Background] (0,-.15) circle (.15);
}
},
flowresistance/.style 2 args={
component,
minimum height = 1cm,
minimum width = 1.5cm,
label={#2:#1},
path picture={%
\draw[hydrline] (-.5,0)--(.5,0);
\draw (1/3+.1,.1) -- (1/3-.1,-.1)
(1/3-.1,+.1) -- (1/3+.1,-.1);
\draw (-1/3+.1,.1) -- (-1/3-.1,-.1)
(-1/3-.1,+.1) -- (-1/3+.1,-.1);
\draw (.1,.1) -- (-.1,-.1)
(-.1,+.1) -- (.1,-.1);
}
},
hydrinertia/.style 2 args={
component,
minimum height = 1cm,
minimum width = 1.5cm,
label={#2:#1},
path picture={%
\draw[hydrline] (-.5,0)--(.5,0);
\fill[Background!80!black] (-.4,-.1) rectangle (.4,.1);
}
},
battery/.style={
line component,
minimum height = 1cm,
minimum width = 6mm+4\pgflinewidth,
path picture={%
\fill[Background] (-3mm,-0.6*\pgflinewidth) rectangle (3mm,0.6*\pgflinewidth);
\draw (3mm,-5mm) -- ++(0,10mm);
\draw (-3mm/3,-5mm) -- ++(0,10mm);
\pgfmathparse{min(2.5*\pgflinewidth,5pt)}
\begin{scope}[line width = \pgfmathresult pt]
\draw (3mm/3,-2.5mm) -- ++(0,5mm);
\draw (-3mm,-2.5mm) -- ++(0,5mm);
\end{scope}
},
},
voltage source/.style={
line component,
minimum height = 1cm,
minimum width=1cm,
path picture={%
\fill[Background] (-0.4,-0.6*\pgflinewidth) rectangle (0.4,0.6*\pgflinewidth);
\draw (0,0) circle (.4);
\draw (-.4,0) -- (.4,0);
\node[path picture={
\draw(0,-.75mm) -- (0,.75mm);
\draw(-.75mm,0) -- (.75mm,0);
}] at (.4,.4){};
\node[path picture={
\draw(-.75mm,0) -- (.75mm,0);
}] at (-.4,.4){};
},
},
currentsensor/.style args={#1}{
line component,
minimum height = 1cm,
minimum width=1cm,
path picture={%
\draw[fill=Background] (0,0) circle (.4);
\node {A};
},
#1,
},
current/.style args={#1}{
line component,
minimum height = 2mm,
minimum width = 2mm,
path picture = {
\fill (-1mm,1mm) -- (1mm,0) -- (-1mm,-1mm) -- cycle;
},
#1,
},
voltagesensor/.style args={#1}{
line component,
minimum height = 1cm,
minimum width=1cm,
path picture={%
\draw[fill=Background] (0,0) circle (.4);
\node {V};
},
#1,
},
oscilloscope/.style args={#1}{
node component,
minimum height = 1cm,
minimum width=1cm,
path picture={%
\draw[fill=Background] (0,0) circle (.4);
\draw (-.25,-.1) -- (0,.1) -- (0,-.1) -- (.25,.1) -- (.25,-.1);
},
#1,
},
functiongenerator/.style args={#1}{
node component,
minimum height = 1cm,
minimum width=1cm,
path picture={%
\draw[fill=Background] (0,0) circle (.4);
\draw[line width = 1pt] (-0.3,0) arc (180:0:.15) arc (-180:0:.15);
},
#1,
},
gyrator/.style={
component,
minimum height=10mm+\pgflinewidth,
minimum width=8mm+\pgflinewidth,
path picture={%
\draw[fill=Background] (-.2,-.5) rectangle (.2,.5);
\draw[fill=Background] (0,0) circle (.4);
}
},
resistor/.style={
line component,
minimum height=4mm+\pgflinewidth,
minimum width=1cm+\pgflinewidth,
path picture={%
\fill[Background] (-0.5,-0.6*\pgflinewidth) rectangle (0.5,0.6*\pgflinewidth);
\draw (-.5,-.2) rectangle (.5,.2);
},
},
resistor2/.style={
line component,
minimum height = 4mm,
minimum width = 1.5cm,
path picture = {
\fill[Background] (-0.5,-0.6*\pgflinewidth) rectangle (0.5,0.6*\pgflinewidth);
\draw[decoration={
aspect=.4,
segment length=2mm,
amplitude=1.5mm,
zigzag,
pre length=1mm,
post length=1mm,
},decorate] (-.5,0) -- (.5,0);
},
},
inductor/.style={
line component,
minimum height=4mm,
minimum width=10mm,
path picture={%
\fill[Background] (-0.5,-0.6*\pgflinewidth) rectangle (0.5,0.6*\pgflinewidth);
\draw (-5mm,0) -- (-4mm,0) arc (180:0:1mm) -- ++(0,-0.5*\pgflinewidth)
(-2mm,0) arc (180:0:1mm) -- ++(0,-0.5*\pgflinewidth)
(0,0) arc (180:0:1mm) -- ++(0,-0.5*\pgflinewidth)
( 2mm,0) arc (180:0:1mm) -- (5mm,0);
},
},
inductor2/.style args={#1}{
line width = 0.4pt,
sloped,
allow upside down,
pos = .5,
minimum height=1cm,
minimum width=1.5cm,
path picture={%
\pgfmathsetmacro{\extraangle}{40}
\pgfmathsetmacro{\radius}{.8/(2+6*cos(\extraangle))}
\fill[Background] (-5mm+.5pt,-.1) rectangle (5mm-.5pt,.1);
\draw (-5mm,0)-- (-4mm,0)
arc (180:-\extraangle:\radius)
arc (180+\extraangle:-\extraangle:\radius)
arc (180+\extraangle:-\extraangle:\radius)
arc (180+\extraangle:0:\radius)
-- (5mm,0);
},
#1
},
capacitor/.style={
line component,
minimum height = 0.8cm,
minimum width = 0.5cm,
path picture={%
\pgfmathparse{max(1.5*\pgflinewidth,2pt)}
\pgfmathsetlengthmacro{\halfwidth}{\pgfmathresult pt}
\fill[Background] (-\halfwidth,-0.6*\pgflinewidth) rectangle (\halfwidth,0.6*\pgflinewidth);
\draw (-\halfwidth,-0.3) -- ++(0,0.6);
\draw (\halfwidth,-0.3) -- ++(0,0.6);
},
},
switch/.style args={#1}{
line component,
minimum height = 7mm,
minimum width = 15mm,
path picture={%
\draw[Background,line width=3pt] (-.4,0) -- (.4,0);
\fill (-.4,0) circle (.5mm);
\fill (.4,0) circle (.5mm);
\draw (-.4,0) -- ++(25:.8);
},
#1
},
electric earth/.style={
component,
minimum height = 1cm,
minimum width=1.5cm,
path picture={%
\draw (0,0) --++(0,-.1)
(-.3,-.1) -- (.3,-.1)
(-.2,-.2) -- (.2,-.2)
(-.1,-.3) -- (.1,-.3);
}
},
mechanical earth/.style args={#1}{
node component,
minimum width = .5cm,
path picture={%
\fill[pattern = north east lines] (path picture bounding box.north west) rectangle (path picture bounding box.south);
\draw(path picture bounding box.north) -- (path picture bounding box.south);
},
#1
},
spring/.style args={#1}{
line component,
minimum height = 0.5cm,
minimum width=1.5cm,
path picture={%
\fill[Background] (-0.5,-0.6*\pgflinewidth) rectangle (0.5,0.6*\pgflinewidth);
\draw(-0.5,0) -- (-0.4,0) decorate[decoration={aspect=.5,segment length=1mm,amplitude=1mm,coil}]{ -- (0.4,0)} -- (0.5,0);
},
#1
},
spring2/.style args={#1}{
resistor2={#1},
},
damper/.style={
line component,
minimum height = 0.5cm,
minimum width=4mm,
path picture={%
\pgfmathparse{max(4pt,2.5*\pgflinewidth)}
\pgfmathsetlengthmacro{\iconix@damper@width}{\pgfmathresult pt}
\fill[Background] (-0.5*\iconix@damper@width,-0.6*\pgflinewidth) rectangle (0,0.6*\pgflinewidth);
\draw (0.5*\iconix@damper@width,-0.2) -- (-0.5*\iconix@damper@width,-0.2) -- (-0.5*\iconix@damper@width,0.2) -- (0.5*\iconix@damper@width,0.2);
\draw (0,0.125) -- (0,-0.125);
},
},
friction/.style 2 args={
line component,
minimum height = 1cm,
minimum width = 1.5cm,
label={[label distance = 0cm]#2:#1},
path picture={%
\fill[pattern = north east lines] (-.5,1/3) rectangle (.5,1/2);
\fill[pattern = north east lines] (-.5,-1/3) rectangle (.5,-1/2);
\draw (-.5,1/3) -- (.5,1/3);
\draw (-.5,-1/3) -- (.5,-1/3);
\draw (-1/3+.1,1/6+.1) -- (-1/3-.1,1/6-.1)
(-1/3-.1,1/6+.1) -- (-1/3+.1,1/6-.1);
\draw (.1,1/6+.1) -- (-.1,1/6-.1)
(-.1,1/6+.1) -- (.1,1/6-.1);
\draw (1/3+.1,1/6+.1) -- (1/3-.1,1/6-.1)
(1/3-.1,1/6+.1) -- (1/3+.1,1/6-.1);
\draw (-1/3+.1,-1/6+.1) -- (-1/3-.1,-1/6-.1)
(-1/3-.1,-1/6+.1) -- (-1/3+.1,-1/6-.1);
\draw (.1,-1/6+.1) -- (-.1,-1/6-.1)
(-.1,-1/6+.1) -- (.1,-1/6-.1);
\draw (1/3+.1,-1/6+.1) -- (1/3-.1,-1/6-.1)
(1/3-.1,-1/6+.1) -- (1/3+.1,-1/6-.1);
}
},
mass/.style args={#1}{
node component,
minimum height = 1.1cm,
minimum width = 1.5cm,
path picture={%
\draw[fill=Background,#1] (-.5,-.5) rectangle (.5,.5);
},
},
mass2/.style args={#1}{
node component,
minimum height = 1.1cm,
minimum width = 1.5cm,
draw,
path picture={%
\node{$ m $};
}
},
inertia/.style={
component,
minimum height = 14mm+\pgflinewidth,
minimum width = 4mm+\pgflinewidth,
path picture={%
\draw[fill=white!80!black] (-.2,-.7) rectangle (.2,.7);
}
},
inertia2/.style 2 args={
mass={#1}{#2},
path picture={%
\draw[fill=Background] (-.5,-.5) rectangle (.5,.5);
\node{$ J $};
}
},
rpossource/.style 2 args={
component,
minimum height = 1cm,
minimum width = 1.5cm,
label={#2:#1},
path picture={%
\draw[line width = 1mm,Background, shorten > = .5pt, shorten < = .5pt] (-.75,0) -- (.75,0);
\draw[<<->>,>=stealth, shorten > = .5pt, shorten < = .5pt] (-.75,0) -- (.75,0);
\node[fill=Background,minimum width = .4cm]{$ \varphi $};
}
},
tpossource/.style 2 args={
component,
minimum height = 1cm,
minimum width = 1.5cm,
label={#2:#1},
path picture={%
\draw[line width = 1mm,Background, shorten > = .5pt, shorten < = .5pt] (-.75,0) -- (.75,0);
\draw[<->,>=stealth, shorten > = .5pt, shorten < = .5pt] (-.75,0) -- (.75,0);
\node[fill=Background,minimum width = .4cm]{$ x $};
}
},
transmissionl1/.style 2 args={
component,
yshift=-1cm,
minimum height = 2.5cm,
minimum width = 1cm,
label={#2:#1},pos=1,
path picture={%
\draw[fill=Background] (0,.1) arc (90:-90:.1);
\draw (0,1) -- ++ (0,-2);
\draw (0,-1) -- ++ (-.25,.125) -- ++ (0, -.25) -- cycle;
\draw (-.25,-.8) -- ++(0,-.4);
\fill[pattern=north east lines] (-.25,-.8) rectangle ++(-.15,-.4);
\draw[fill=Background] (0,1) circle (.1);
\draw[fill=Background] (0,-1) circle (.1);
}
},
transmissiong1/.style 2 args={
component,
yshift=-1cm,
minimum height = 2.5cm,
minimum width = 1cm,
label={#2:#1},
pos=0,
path picture={%
\draw[fill=Background] (0,.1) arc (-270:-90:.1);
\draw (0,1) -- ++ (0,-2);
\draw (0,-1) -- ++ (-.25,.125) -- ++ (0, -.25) -- cycle;
\draw (-.25,-.8) -- ++(0,-.4);
\fill[pattern=north east lines] (-.25,-.8) rectangle ++(-.15,-.4);
\draw[fill=Background] (0,1) circle (.1);
\draw[fill=Background] (0,-1) circle (.1);
}
},
cabledrumr/.style 2 args={
component,
yshift=-.5cm,
minimum height = 2cm,
minimum width = 1.5cm,
label={#2:#1},
pos = 1,
path picture={%
\draw[fill=Background] (-.3,.6) rectangle (.3,-.5);
\draw (-.3,.5) --++ (.6,-.05)
++(-.6,-.05) --++ (.6,-.05)
++(-.6,-.05) --++ (.6,-.05)
++(-.6,-.05) --++ (.6,-.05);
}
},
cabledruml/.style 2 args={
component,
yshift=-.5cm,
minimum height = 2cm,
minimum width = 1.5cm,
label={#2:#1},
pos = 1,
path picture={%
\draw[fill=Background] (-.3,.6) rectangle (.3,-.5);
\draw (.3,.5) --++ (-.6,-.05)
++(.6,-.05) --++ (-.6,-.05)
++(.6,-.05) --++ (-.6,-.05)
++(.6,-.05) --++ (-.6,-.05);
}
},
forcesource/.style 2 args={
line component,
minimum height = 1cm,
minimum width = 1.5cm,
label={#2:#1},
path picture={%
\draw (-0.2,.15) -- (0.2,0) -- (-0.2,-.15);
}
},
torquesource/.style 2 args={
component,
minimum height = 1cm,
minimum width = 1.5cm,
label={#2:#1},
path picture={%
\draw (-0.35,.15) -- (0.05,0) -- (-0.35,-.15);
\draw (-0.05,.15) -- (0.35,0) -- (-0.05,-.15);
}
},
forcesource2/.style args={#1}{
line component,
minimum height = 0.35cm,
minimum width = 1.5cm,
path picture={%
\fill[Background] (-7.5mm+0.5\pgflinewidth,-0.6*\pgflinewidth) rectangle (7.5mm-0.5*\pgflinewidth,0.6*\pgflinewidth);
\node[minimum width = .4cm,minimum height = .3cm](F){$ F $};
\begin{scope}[-stealth, shorten > = 0.5\pgflinewidth]
\draw (F) -- (path picture bounding box.west);
\draw (F) -- (path picture bounding box.east);
\end{scope}
},
#1
},
torquesource2/.style 2 args={
component,
minimum height = 1cm,
minimum width = 1.5cm,
label={#2:#1},
path picture={%
\draw[line width = 1mm,Background, shorten > = .5pt, shorten < = .5pt] (-.75,0) -- (.75,0);
\draw[<<->>,>=stealth, shorten > = .5pt, shorten < = .5pt] (-.75,0) -- (.75,0);
\node[fill=Background,minimum width = .4cm]{$ T $};
}
},
centerofmass/.style={
line width = 1pt,
minimum size = #1+1pt,
circle,
draw,
path picture={%
\draw[black,line width=1pt] (0,0) circle (.5*#1-.5pt);
\fill[Background] (#1,#1) rectangle (-#1,-#1);
\fill[] (0,0) rectangle (-#1,-#1);
\fill[] (0,0) rectangle (#1,#1);
}
},
}
\tikzset{
    numbered grid label/.code={
        \tikzset{
            gray,
            font=\tiny,
            inner ysep=0.5pt,
            inner xsep=2.5pt,
        }
        \pgfkeys{
            /pgf/number format/.cd,
                fixed,
                fixed zerofill=true,
                precision=3,
        }
    },
}

\def\numgrid{\@ifnextchar[%
    {\numgrid@i}{\numgrid@i[1]}%
}
\def\numgrid@i[#1]{\@ifnextchar[%
    {\numgrid@ii{#1}}{\numgrid@ii{#1}[0]}%
}
\def\numgrid@ii#1[#2]{\@ifnextchar[%
    {\numgrid@iii{#1}{#2}}{\numgrid@iii{#1}{#2}[0]}
}
\def\numgrid@iii#1#2[#3]#4#5{
    \pgfmathsetmacro\xstep{#1+#2}%
    \pgfmathsetmacro\ystep{#1+#3}%
    \foreach \x in {#2,\xstep,...,#4}{
        \draw[help lines] (\x,#3) -- (\x,#5) foreach \pos/\place in {0/left,1/right}{
            node[numbered grid label,pos=\pos, \place,rotate=90]{\pgfmathprintnumber{\x}}};
    }
    \foreach \y in {#3,\ystep,...,#5}{
        \draw[help lines] (#2,\y) -- (#4,\y) foreach \pos/\place in {0/left,1/right}{
            node[numbered grid label,pos=\pos, \place]{\pgfmathprintnumber{\y}}};
    }
}
\makeatother
\endinput
%%
%% End of file `xbg.code.iconicdiagrams.tex'.
